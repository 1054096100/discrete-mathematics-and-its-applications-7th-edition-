###1.1 练习题答案

[TOC]

####1.下列哪些语句是命题?这些是命题的语句的真值是什么?

#####a) 波士顿是马赛诸塞州首府。
>T

#####b) 迈阿密是佛罗里达州首府。
>F

#####c) 2 + 3 = 5。
>T

#####d) 5 + 7 = 10。
>F

#####e) x + 2 = 11。
>不是命题。(因为这个语句既真又假。)

#####f) 回答这个问题。
>不是命题。(应该是因为这是一个祈使句,不是陈述句。)

####2. 下列哪些是命题?这些命题的真值是什么?

#####a) 别过去。
>不是命题(个人认为是祈使句,不是陈述句。)。无真值。

#####b) 几点了?
>不是命题(个人认为是疑问句,不是陈述句。)。无真值。

#####c) 在缅因州没有黑苍蝇。
>是命题。真值为假。(这里的黑苍蝇应该是指黑尾黑麻蝇,黑尾黑麻蝇分布于全北界。)

#####d) 4 + x = 5。
>不是命题(因为这个语句既真又假。)。无真值。

#####e) 月亮是由绿色的奶酪构成的。
>是命题。真值为假。(月球是固态卫星。组成包括月壤、玻璃颗粒和月岩。)

#####f) 2^n^ $\geq$ 100。
>不是命题(因为这个语句既真又假。)。无真值。

####3. 下列各命题的否定是什么?[^p.s.]

#####a) Mei 有一台 MP3 播放器。
>Mei 没有一台 MP3 播放器。

#####b) 新泽西没有污染。
>新泽西有污染。

#####c) 2 + 1 = 3。
>2 + 1 $\neq$ 3。

#####d) 缅因州的夏天又晒又热。
>缅因州的夏天晒和热的情况不会同时发生。

####4. 下列各命题的否定是什么?[^p.s.]

#####a) Jennifer 和 Teja 是朋友。

>Jennifer 和 Teja 不是朋友。

#####b) 面包师说一打有十三个。
>面包师没有说一打有十三个。

#####c) Abby 每天发送 100 多条文本信息。
>Abby 每天不会发送 100 多条文本信息。

#####d) 121 是一个完全平方数。
>121 不是一个完全平方数。

####5. 下列命题的否定是什么?[^p.s.]

#####a) Steve 的笔记本电脑有大于 100 GB 的空闲磁盘空间。
>Steve 的笔记本电脑没有大于 100 GB 的空闲磁盘空间。

#####b) Zach 阻止来自 Jennifer 的邮件与短信。
>Zach 接受了来自 Jennifer 的邮件或短信。

#####c) 7 $\cdot$ 11 $\cdot$ 13 = 999。
>7 $\cdot$ 11 $\cdot$ 13 $\neq$ 999。

#####d) Diane 周日骑自行车骑了 100 英里。
>Diane 没有在周日骑自行车骑了 100 英里。

####6. 假设智能手机 A 有 256 MB RAM 和 32 GB ROM,并且其照相机的分辨率是 8 MP;智能手机 B 有 288 MB RAM 和 64 GB ROM, 并且其照相机的分辨率是 4 MP;而智能手机 C 有 128 MB RAM 和 32 GB ROM, 并且其照相机的分辨率是 5 MP。试判定下面每个命题的真值。

#####a) 智能手机 B 的 RAM 是三款手机中最多的。
>T

#####b) 智能手机 C 比智能手机 B 具有更多的 ROM 或者更高分辨率的相机。
>T

#####c) 智能手机 B 比智能手机 A 具有更多的 RAM、更多的 ROM 和更高分辨率的相机。
>F

#####d) 如果智能手机 B 比智能手机 C 具有更多的 RAM 和更多的 ROM,则它也具有更高分辨率的相机。

>F

#####e) 智能手机 A 比智能手机 B 具有更多的 RAM 当且仅当智能手机 B 比智能手机 A 具有更多的 RAM。
>F

####7. 假设在最近的财年期间, Acme 计算机公司的年收入是 1380 亿美元且其净利润是 80 亿美元, Nadir 软件公司的年收入是 870 亿美元且净利润是 50 亿美元, Quixote 媒体的年收入是 1110 亿美元且净利润是 130 亿美元。试判断有关最近财年的每个命题的真值。

#####a) Quixote 媒体的年收入最多。
>F

#####b) Nadir 软件公司的净利润最少并且 Acme 计算机公司的年收入最多。
>T

#####c) Acme 计算机公司的净利润最多或者 Quixote 媒体的净利润最多。
>T

#####d) 如果 Quixote 媒体的净利润最少,则 Acme 计算机公司的年收入最多。
>T

#####e) Nadir 软件公司的净利润最少当且仅当 Acme 计算机公司的年收入最多。
>T

####8. $令\ p\ 、\ q\ 为如下命题:\\p\ :本周我买了一张彩票。\\q\ :我赢得了百万美元大奖。\\试用汉语表达下列各命题。$

#####a) $\neg p$
>本周我没有买了一张彩票。[^p.s.]

#####b) $p \vee q$
>本周我买了一张彩票,或者赢得了百万美元大奖,或者两者都成。

#####c) $p \to q$
>只要本周我买了一张彩票,就会赢得了百万美元大奖。

#####d) $p \wedge q$
>本周我买了一张彩票,并且赢得了百万美元大奖。

#####e) $p \leftrightarrow q$
>本周我买了一张彩票当且仅当赢得了百万美元大奖的时候。
>*(等价于)*
>我赢得了百万美元大奖当且仅当本周买了一张彩票。

#####f) $\neg p \to \neg q$
>如果本周我没买了一张彩票,那么就不会赢得了百万美元大奖。

#####g) $\neg p \vee \neg q$
>本周我没有买了一张彩票,也没有赢得了百万美元大奖。

#####h) $\neg p\vee (p \wedge q)$
>本周我没有买一张彩票,兼或本周我买了一张彩票且赢得了百万美元大奖。
>*(这是按照逻辑语法强译过来的,看起来不像是人说出来的话,我也不太会改,所以先凑活着吧,也求哪位 dalao 相助。)*

####9. 令 $p$ 和 $q$ 分别表示命题“在新泽西州海岸游泳是允许的”和“在海岸附近发现过鲨鱼”。试用汉语表达下列每个复合命题。

#####a) $\neg q$
>在海岸附近没有发现过鲨鱼。[^p.s.]

#####b) $p\wedge q$
>在新泽西州海岸游泳是允许的且在海岸附近发现过鲨鱼。

#####c) $\neg p\vee q$
>在新泽西州海岸游泳是不被允许的,兼或在海岸附近发现过鲨鱼。
>*(这是按照逻辑语法强译过来的,看起来不像是人说出来的话,所以我改了一下。)*
>在新泽西海岸附近没有发现鲨鱼的情况下,不会出现在海岸游泳的行为被允许的情况。
>*(看上去还是不太像人话,不过我尽力了,有思路的 dalao 帮忙改改吧。。。)*

#####d) $p\to\neg q$
>如果在新泽西州海岸游泳是允许的,那么在海岸附近没有发现过鲨鱼。

#####e) $\neg q \to p$
>如果在海岸附近没有发现过鲨鱼,那么在新泽西州海岸游泳是允许的。

#####f) $\neg p \to \neg q$
>如果在新泽西州海岸游泳是不被允许的,那么在海岸附近没有发现过鲨鱼。

#####g) $p\leftrightarrow \neg q$
>在新泽西州海岸附近没有发现过鲨鱼当且仅当在海岸游泳是允许的。
>*(等价于)*
>在新泽西州海岸游泳是允许的当且仅当在海岸附近没有发现过鲨鱼。

#####h) $\neg p \wedge (p\vee\neg q)$
>在新泽西州海岸游泳是不被允许的情况与在海岸附近没有发现过鲨鱼兼或在新泽西州海岸游泳是允许的的情况要同时发生。
>*(这是按照逻辑语法强译过来的,看起来不像是人说出来的话,所以我改了一下。)*
>在新泽西州海岸游泳是不被允许的,而且如果在新泽西海岸附近发现过鲨鱼,则在新泽西州海岸游泳是允许的。
>*(看上去还是不太像人话,不过我尽力了,有思路的 dalao 帮忙改改吧。。。)*

####10. 令 $p$ 和 $q$ 分别表示命题“选举已经有了结果”和“选票已经计数完毕”。 试用汉语表达下列各命题。

#####a) $\neg p$
>选举还没有出结果[^p.s.]。

#####b) $p \vee q$
>选举已经有了结果,或者选票已经计数完毕,或者两者都已经结束。

#####c) $\neg p \wedge q$
>选举还没有出结果,但是选票已经计数完毕。

#####d) $q \to p$
>如果选票已经计数完毕,那么选举已经有了结果。

#####e) $\neg q \to \neg p$
>如果选举还没有出结果,那么选票还没有计数完毕。

#####f) $\neg p \to \neg q$
>如果选举还没有出结果,则是选票还没计数完毕。

#####g) $p\leftrightarrow q$
>选举已经有了结果当且仅当选票已经计数完毕。
>*(等价于)*
>选票已经计数完毕当且仅当选举已经有了结果。

#####h) $\neg q \vee (\neg p \wedge q)$
>选票还没计数完毕,兼或选举还没有出结果且选票已经计数完毕。
>*(强译的,看着慢慢觉得就像是人话了,我可能是疯了,自我救赎一下。)
>要么是选票还没计数完毕,要么是出现了选举还没有出结果但选票已经计数完毕的情况,要么以上两种情况都发生了。
>(我可能是疯了会觉得这个还不错。虽然完全说不通,但是总觉得最后半句不能扔掉。)

####11. $令\  p\ 、\ q\ 为如下命题:\\ p\ :气温在零度以下。\\ q\ :正在下雪。\\  用\ p\ 、\ q\ 和逻辑连接词(包括否定)写出下列各命题:$

#####a) 气温在零度以下且正下着雪。
>$p\wedge q$

#####b) 气温在零度以下,但没下着雪。
>$p \wedge \neg q$

#####c) 气温不在零度以下,并且没有下雪。
>$\neg p \wedge \neg q$

#####d) 也许正下着雪,也气温在零度以下(也许两者兼有)。
>$q\vee p$

#####e) 如果气温在零度以下,则没有下雪。
>$p\to q$

#####f) 也许气温在零度以下,也许下着雪;但是如果在零度以下,就没有下雪。
>$(p \vee q)\wedge(p \to \neg q)$

#####g) 气温在零度以下是下雪的充分必要条件。
>$p \leftrightarrow q$

####12. $令\ p\ 、\ q\ 和\ r\ 为如下命题:\\p\ :你得流感了。\\q\ :你错过了期终考试。\\r\ :这门课你及格了。\\将下列各命题用汉语表示:$

#####a) $p \to q$
>如果你得流感了,你就错过了期终考试。

#####b) $\neg q \leftrightarrow r$
>你没有错过期终考试当且仅当这门课你及格了。
>*(等价于)*
>这门课你及格了当且仅当你没有错过期终考试。

#####c) $q\to\neg r$
>如果你错过了期终考试,这门课你就不会及格了。

#####d) $p \vee q\vee r$
>要么你得流感了,要么你错过了期终考试,要么这门课你及格了(三种情况互容且可共存)。

#####e) $(p \to \neg r)\vee(q \to \neg r)$
>要么因为你得流感这门课你会不及格,要么因为错过期终考试这门课你会不及格,要么都是。

#####f) $(p\wedge q)\vee(\neg q\wedge r)$
>要么你得流感了并错过了期终考试,要么你没有错过期终考试而且这门课你及格了,哟啊么两种情况共存。

####13. $令\ p\ 、\ q\ 为如下命题:\\ \ p\ :你的车速超过每小时\ 65\ 英里(1\ 英里\ =\ 1.6\ 公里)\\ \ q\ :你接到一张超速罚单。\\ 用\ p\ 、\ q\ 和逻辑连接词(包括否定)写出下列命题。$

#####a) 你的车速没有超过每小时 65 英里。
>$\neg p$

#####b) 你的车速超过每小时 65 英里,但是没有接到超速罚单。
>$p \wedge \neg q$

#####c) 如果你的车速超过每小时 65 英里,你就会接到一张超速罚单。
>$p \to q$

#####d) 如果你的车速没有超过每小时 65 英里,你就不会接到超速罚单。
>$\neg p \to\neg q$

#####e) 车速超过每小时 65 英里足以接到超速罚单。
>$p \to q$

#####f) 你接到一张超速罚单,但是你的车速没有超过每小时 65 英里。
> $q \wedge \neg p$

#####g) 只要你接到一张超速罚单,你的车速就超过了每小时 65 英里。
>$q \to p$

####14. $令\ p\ 、\ q\ 、\ r\ 为如下命题:\\p\ :你的期末考试得了\ A\ 。\\q\ :你做了本书每一道练习。\\r\ :这门课你得了\ A\ 。\\用\ p\ 、\ q\、\ r\和逻辑联结词(包括否定)写出下列命题:$

#####a) 这门课你得了 A,但你并没做本书的每道练习。
>$r\wedge \neg q$

#####b) 你的期末考试得了A,你做了本书的每一道练习,并且这门课你得了 A。
>$p \wedge q\wedge r$

#####c) 想在这门课得 A,你必须在期末考试得 A。
>$p \to r$

#####d) 你的期末考试得了 A,你没有做本书的每道练习;然而这门课你还是得了 A。
>$p \wedge \neg q \wedge r$

#####e) 期末考试得 A 并且做本书的每道练习, 足以使你这门课得 A。
>$(p \wedge q)\to r$

#####f) 这门课得 A 当且仅当你做本书的每道练习或期末考试得 A。
>$r \leftrightarrow (q \vee p)$

####15. $令\ p\ 、\ q\ 、\ r\ 为如下命题:\\ p\ :在这个地区发现过灰熊。\\ q\ : 在乡间道路上徒步旅行是安全的。\\ r\ :乡间小路两旁的草莓成熟了。\\ 用\ p\ 、\ q\ 、\ r\ 和逻辑联结词(包括否定)写出下列命题:$

#####a) 乡间小路两旁的草莓成熟了,但在这个地方没有发现过灰熊。
>$r \wedge \neg p$

#####b) 在这个地区没有发现过灰熊,且在乡村小路上徒步旅行是安全的,但乡间小路两旁的草莓成熟了。
>$\neg p \wedge q \wedge r$

#####c) 如果乡间小路两旁的草莓成熟了,徒步旅行是安全的当且仅当在这个地区没有发现过灰熊。
>$r\to (q \leftrightarrow\neg p)$

#####d) 在乡间道路上徒步旅行是不安全的,但在这个地区没有发现过灰熊且小路两旁的草莓成熟了。
>$\neg q \wedge \neg p \wedge r$

#####e) 为了使乡间小路上旅行很安全,其必要但非充分条件是乡间小路两旁的草莓没有成熟且在这个地区没有发现过灰熊。
>$q\to (\neg r \wedge \neg p)$

#####f) 无论何时在这个地区发现过灰熊且乡间小路两旁的草莓成熟,在乡间小路上徒步旅行都不安全。
>$(p \wedge r) \to \neg q$

####16. 判断下列各条件语句的真假:

#####a) 2 + 2 = 4 当且仅当 1 + 1  = 2。
>T

#####b) 1 + 1 = 2 当且仅当 2 + 3 = 4。
>F

#####c) 1 + 1 = 3 当且仅当猴子会飞。
>T

#####d) 0 > 1 当且仅当 2 > 1。
>F

####17. 判断下列各条件语句的真假:

#####a) 如果 1 + 1 = 2,则 2 + 2 = 5。
>F

#####b) 如果 1 + 1 = 3,则 2 + 2 = 4。
>T

#####c) 如果 1 + 1 = 3,则 2 + 2 = 5。
>T

#####d) 如果猴子会飞,那么 1 + 1 = 3。
>T

####18. 判断下列各条件语句的真假:

#####a) 如果 1 + 1 = 3,那么独角兽存在。
>T

#####b) 如果 1 + 1 = 3,狗就会飞。
>T

#####c) 如果 1 + 1 = 2,狗就会飞。
>F

#####d) 如果 2 + 2 = 4,那么 1 + 2 = 3。
>T

####19. 下列各语句,判断其中想表达的是兼或还是异或,说明理由。

#####a) 晚饭有咖啡或者茶。
>两个均可,主要看语境了。
>兼或($\vee$)因为晚饭可以两个都点。(饭店点餐)
>异或($\oplus$)因为晚餐的饮料只能点一个。(分配打饭)

#####b) 口令必须至少包含 3 个数字或至少 8 个字符长。
>兼或($\vee$)因为口令可以同时包含 3 个数字且比八个字符长。

#####c) 这门课程的先修课程是数论课程或者密码课程。
>兼或($\vee$)因为口令可以同时先修数论课程和密码课程。

#####d) 你可以用美元或者欧元支付。
>异或($\oplus$),一般人不会又支付美元又支付欧元的吧。。。

####20. 下列各语句,判断其中想表达的是兼或还是异或,说明理由。

#####a) 要求有c++或Java的经验。
>兼或($\vee$)因为可以两个语言都有经验(应该不会有公司限制语言上限吧。。。)。

#####b) b)	午餐包括汤或沙拉。
>兼或($\vee$)因为午餐可以两个都点。(饭店点餐)
>(汤是汤类,沙拉是小吃类应该也不会出现非得选一个不能吃另外一个的情况。)

#####c) 你必须持护照或选民登记卡才能人境。
>兼或($\vee$)因为同时持护照与选民登记卡是可以入境的。

#####d) 出版或销毁。
>异或($\oplus$),因为出版后就不好销毁了,销毁了的东西也不太容易在出版。

####21. 对下列语句,说一说如果其中的连接词是兼或($\vee$)与异或($\oplus$)时的含义。你认为语句想表达的是哪个或?

#####a) 要选修离散数学课程,你必须已经选修了微积分或一门计算机科学的课程。
>兼或($\vee$)表示要选修离散数学,不能同时选修过微积分和任何一门计算机科学的课程。
>异或($\oplus$)表示要选修离散数学,可以同时选修微积分和某一计算机课程。
>应选兼或($\vee$)。

#####b) 当你从 Acme 汽车公司购买一部新车时,你就能得到 2000 美元现金折扣或 2% 的汽车贷款。
>兼或($\vee$)表示要你能同时得到 2000 美元现金折扣或 2% 的汽车贷款。
>异或($\oplus$)表示只能得到 2000 美元现金折扣或 2% 的汽车贷款中的一项,不可兼得。
>应选异或($\oplus$)。

#####c) 两人套餐包括 A 栏中的两道菜或 B 栏中的三道菜。
>兼或($\vee$)表示点一份两人套餐可以同时拿到 A 栏的两道菜和 B 栏里的三道菜。
>异或($\oplus$)表示点一份两人套餐只可以拿到 A 栏的两道菜或者 B 栏里的三道菜,不可兼得。
>应选异或($\oplus$)。

#####d) 如果下雪超过两英尺或寒风指数低于 -100 ,学校就停课。
>兼或($\vee$)表示如果两种情况同时达到,学校会停课。
>异或($\oplus$)表示如果这两种情况都达到,学校就不停课。
>应选兼或($\vee$)。

####22. 把下列语句写成“如果 $p$ , 则 $q$ "的形式。(提示: 参考本节列出的条件语句的常用表达方式。)

#####a) 要想晋升, 帮老板洗车是很有必要的。
>如果某人晋升了,则他帮老板洗车了。

#####b) 吹南风预示着春天要来了。
>如果吹南风,则春天要来了。

#####c) 保修单有效的充分条件是你的计算机购买时间不超过一年。
>如果你的计算机购买时间不超过一年,则保修单有效。

#####d) Willy 只要行骗就会被抓住。
>如果 Willy 行骗,则他会被抓住。

#####e) 只有支付了订阅费,你才能访问网站。
>如果你能访问网站,则你支付了订阅费。

#####f) 想要当选必须了解合适的人群。
>如果某人当选了,则他了解了合适的人群。

#####g) 每当坐船 Carol 就会晕船。
>如果 Carol 坐船,则她会晕船。

####23. 把下列语句写成“如果 $p$ ,那么 $q$ "的形式。(提示:参考条件语句的常用表达方式。)

#####a) 只要吹东北风,就会下雪。
>如果吹东北风,则会下雪。

#####b) 苹果树会开花,如果天暖持续一周。
>如果天暖持续一周,则苹果树开花。

#####c) 活塞队赢得冠军就意味着他们打败了湖人队。
>如果活塞队赢了冠军,则他们打败了湖人队。

#####d) 必须走 8 英里才能到达朗斯峰的顶峰。
>如果你登上了达朗斯峰,则你走了 8 英里。

#####e) 想要得到终身教授职位,只要能世界闻名就够了。
>如果某人能世界闻名,则他能得到终身教授职位。

#####f) 如果你驾车超过 400 英里,就需要买汽油了。
>如果你驾车超过 400 英里,则你需要买汽油了。

#####g) 你的保修单是有效的,只有当你购买的 CD 机不超过 90 天。
>如果你的保修单是有效的,则你购买的 CD 机不超过 90 天。

#####h) Jan 要去游泳,除非水太凉了。
>如果 Jan 没去游泳,则水太凉了。

####24. 把下列语句写成“如果 $p$ ,那么 $q$ "的形式。(提示:参考本节列出的条件语句的常用表达方式。)

#####a) 我会记得把地址发给你,仅当你给我发一封电子邮件。
>如果我记得把地址发给你,则你给我发一封电子邮件。

#####b) 要成为美国公民,只要你生在美国就行了。
>如果你生在美国,则能成为美国公民。

#####c)	如果你保存好课本,它会是你未来其他课程有用的参考书。
>如果你保存好课本,则它会是你未来其他课程有用的参考书。

#####d)	红翼队将赢得斯坦利杯,如果其守门员表现出色。
>如果红翼队的守门员表现出色,则他们将赢得斯坦利杯。

#####e)	你获得这一职位,表明你有最好的信誉。
>如果你获得这一职位,则你有最好的信誉。

#####f)	有风暴时沙滩会受到侵蚀。
>如果有沙暴,则沙滩会受到侵蚀。

#####g)	必须有一个有效的口令才能登录到服务器。
>如果登录到了服务器,则有一个有效的口令。

#####h)	你能登顶,除非你太晚才开始爬山。
>如果你太晚才开始爬山,则你不能登顶。

####25) 把下列命题写成“ $p$ 当且仅当 $q$ ”的形式。

#####a) 如果外边热你就买冰激凌蛋卷;并且如果你买冰激凌蛋卷,外边就热。
>外面热当且仅当你买冰淇淋蛋卷。
>*(等价于)*
>你买冰淇淋蛋卷当且仅当外面热。

#####b) 你赢得竞赛的充分必要条件是你有唯一的获胜券。
>你赢得竞赛当且仅当你有唯一的优胜券。
>*(等价于)*
>你有唯一的优胜券当且仅当你赢得竞赛。

#####c) 你能得到提拔,只有当你有关系网;并且你有关系网,只有当你得到了提拔。
>你能得到提拔当且仅当只有你有关系网。
>*(等价于)*
>只有你有关系网当且仅当你能得到提拔。

#####d) 如果你看电视,心智会衰退;反之亦然。
>你看电视当且仅当你心智会衰退。
>*(等价于)*
>你的心智会衰退当且仅当你看电视。

#####e) 火车恰恰在我乘坐的那些日子晚点。
>火车晚点当且仅当我那天要乘坐它。
>*(等价于)*
>我乘坐火车当且仅当它当天会晚点。

####26. 把下列命题写成“ $p$ 当且仅当 $q$ ”的形式。

#####a) 你能在这门课得 A 的充分必要条件是学会解离散数学问题。
>你能在这门课得 A 当且仅当你学会解离散数学问题。
>*(等价于)*
>你学会解离散数学问题当且仅当你能在这门课得 A 。

#####b) 如果你每天看报,你就了解情况;反之亦然。
>你每天看报当且仅当你了解情况。
>*(等价于)*
>你了解情况当且仅当你每天看报。

#####c) 如果是周末,天就下雨;如果天下雨,就是周末。
>某一天是周末当且仅当当天下雨。
>*(等价于)*
>某天下雨当且仅当当天是周末。

#####d) 仅当巫师不在家时你能看到巫师,仅当你能看到巫师时巫师不在家。
>巫师不在家当且仅当你能看到巫师。
>*(等价于)*
>你能看到巫师当且仅当巫师不在家。

####27. 给出下列各条件语句的逆命题、逆否命题和反命题。
#####a) 如果今天下雪,我明天就去滑雪。
>逆命题:如果我明天去滑雪,则今天下雪。
>逆否命题:如果我明天没去滑雪,则今天没下雪。
>反命题:如果今天没下雪,我明天就不去滑雪

#####b) 只要有测验,我就来上课。
>逆命题:如果我来上课,则有测验。
>逆否命题:如果我没来上课,则没有测验。
>反命题:如果没有测验,我就来不上课。

#####c) 一个正整数是素数,仅当它没有 1 和自身以外的因子。
>逆命题:一个正整数没有 1 和自身以外的因子,仅当他是素数。
>逆否命题:一个正整数有 1 和自身以外的因子,仅当他不是素数。
>反命题:一个正整数不是素数,仅当它有 1 和自身以外的因子。

####28. 给出下列条件语句关系的逆命题、逆否命题和反命题。

#####a) 如果今晚下雪,我将待在家里。
>逆命题:如果我今晚待在家里,则今晚下雪了。
>逆否命题:如果我今晚没有待在家里,则今晚没有下雪。
>反命题:如果今晚没有下雪,我将不会待在家里。

#####b) 只要是阳光充足的夏天,我就去海滩。
>逆命题:只要是我去海滩,就是阳光充足的夏天。
>逆否命题:只要是我没去海滩,就不是阳光充足的夏天。
>反命题:只要不是阳光充足的夏天,我就不去海滩。

#####c) 如果我工作到很晚,那我就有必要睡到中午。
>逆命题:如果我有必要睡到中午,那我有工作到很晚。
>逆否命题:如果我没有必要睡到中午,那我没有工作到很晚。
>反命题:如果我没有工作到很晚,那我就没有必要睡到中午。

####29. 下列各复合命题的真值表会有多少行?

#####a) $p \to \neg p$
>2

#####b) $(p\vee \neg r)\wedge (q \vee \neg s)$
>16

#####c) $q \vee p \vee \neg s \vee \neg r \vee \neg t \vee u$
>64

#####d) $(p \wedge r \wedge t)\leftrightarrow (q \wedge t)$
>16

####30. 下列各复合命题的真值表会有多少行?

#####a) $(q \to \neg p)\vee(\neg p \to \neg q)$
>4

#####b) $(p \vee \neg t)\wedge (p \vee \neg s)$
>8

#####c) $(p \to r)\vee(\neg s \to \neg t)\vee(\neg u \to v)$
>64

#####d) $(p \wedge r \wedge s)\vee(q \wedge t)\vee(r \wedge \neg t)$
>32

####31. 构造下列各复合命题的真值表。

#####a) $p \wedge \neg p$
|$p$|$p \wedge \neg p$|
|---|---|
|T|F|
|F|F|
#####b) $p \vee \neg p$
|$p$|$p \vee \neg p$|
|---|-|
|T|T|
|F|T|
#####c) $(p \vee \neg q)\to q$
|$p$|$q$|$(p \vee \neg q)\to q$|
|---|||
|T|T|T|
|T|F|F|
|F|T|T|
|F|F|F|
#####d) $(p \vee q)\to(p \wedge q)$

#####e) $(p \to q)\leftrightarrow (\neg q \to \neg p)$

#####f) $(p \to q)\to(q \to p)$






虽然你足够聪明,但是我还是觉得应该哼哼几句。
逻辑规则 是离散数学的基础,请务必理清。
生活中的使用不比教材,会

---
[^p.s.]: 本题是语言上的命题理解和其否定的搜索,所以在语言的理解上可能会有很多种理解,比如对语句的重点的判断和语境的联想,这里只能列出本人的理解和本人认为比较合乎本人理解的否定。)


