###1.1

[TOC]

####1.1.1 少许注意事项

虽然你足够聪明,但是我还是觉得应该哼哼几句。
逻辑规则 是离散数学的基础,请务必理清。
生活中的使用不比教材,会有些许出入,请随机应变。
科学允许质疑,但是有些请先相信。

####1.1.2 命题

**命题**:陈述语句,有**真值**属性,真/假,不能既真又假。
**命题变元**:代表命题的变量(T = True, F = False)。
**命题演算/命题逻辑**:有一个或者多个命题组合成新的命题,有**逻辑运算符**构造,称为**联结词**。
**真值表**:命题的变元和真值的对应。
例如:

| p    | $$\neg p$$ |
| ---- | ---------- |
| T    | F          |
| F    | T          |

**定义1**:
 p为一命题,则p的**否定**记作$$\neg p$$(也可记作$$\bar{p}$$),表示一个与p的真值相反的命题,命题$$\neg p$$读作“非p”。
**定义2**:
 p和q为命题。p、q的**合取**即命题“p且q”,记作$$p \wedge q$$。当p和q都为真时,$$p \wedge q$$ 命题为真,否则为假。
**定义3**:
 p和q为命题。p、q的**析取**即命题“p或q“,又称**”兼或“(inclusive or)**,记作$$p \vee q$$ 。当p和q均为假时,$$p \vee q$$命题为假,否则为真。
**定义4**:
 p和q为命题。p、q的**异或(exclusive or)**即命题”p异或q“,记作$$p \oplus q$$。当p和q的真值互不相同时,$$p \oplus q$$命题为真,否则为假。

示例:

| p    | q    | $$p \wedge q$$ | $$p \vee q$$ | $$p \oplus q$$ |
| ---- | ---- | -------------- | ------------ | -------------- |
| T    | T    | T              | T            | F              |
| T    | F    | F              | T            | T              |
| F    | T    | F              | T            | T              |
| F    | F    | F              | F            | F              |

####1.1.3 条件语句

#####几个主要的命题合成方式:

**定义5**:令p和q为命题。条件语句$$p\to q$$即命题”如果p,则q“。当p为真而q为假时,$$p to q$$为假,否则为真。又称为**蕴含**。
>(p.s.:p为假设(前件、前提)、q为结论(后件))
|p|q|$$p \to q$$|
|---|||
|T|T|T|
|T|F|F|
|F|T|T|
|F|F|T|
>(p.s.:"p仅当q"等同于”如果p,则q“等同于”q除非$$\neg p$$“,一般用”$$p \vee q$$“思考等价式。)

##### 逆命题、逆否命题与反命题

由$$p\to q$$可以衍生出一些新的条件语句。其中常见的:
命题$$q \to p$$称作$$p \to q$$的逆命题;
命题$$\neg p \to \neg  q$$称为$$p\to q$$的否命题;
命题$$\neg q \to \neg p$$称为$$p \to q$$的逆否命题。
|p|q|$$p \to q$$|$$q \to p$$|$$\neg p \to \neg q$$|$$\neg q \to \neg p$$|
|---||||||
|T|T|T|T|T|T|
|T|F|F|T|T|F|
|F|T|T|F|F|T|
|F|F|T|T|T|T|
>(p.s.:当两个命题总是具有相同的真值时,两个命题称为**等价的**[^介绍]。

[^介绍]: 在后续的第五章会详细介绍。

#####双条件语句

**定义6**:
p和q为命题。p、q的**双条件语句**即命题”p当且仅当q“,记作$$p \leftrightarrow q$$。当p和q的真值相同时,$$p \leftrightarrow q$$命题为真,否则为假。双条件语句也称为**双向蕴含**。

|p|q|$$p \leftrightarrow q$$|
|--|||
|T|T|T|
|T|F|F|
|F|T|F|
|F|F|T|

####1.1.4 复合命题的真值表

例如:

|p|q|$$\neg p$$|$$p \vee \neg q$$|$$p \wedge q$$|$$(p \vee \neg q)\to(p \wedge q)$$|
|---|---|---|---|---|---|
|T|T|F|T|T|T|
|T|F|T|T|F|F|
|F|T|F|F|F|T|
|F|F|T|T|F|F|

####1.1.5逻辑运算符的优先级

优先级越高(即数字越小),越先运算[^优先级习惯0]。
|运算符|优先级|
|---|---|
|$$\neg $$|1|
|$$\wedge$$|2|
|$$\vee$$|3[^优先级习惯1]|
|$$\to$$|4|
|$$\leftrightarrow$$|5[^优先级习惯2]|

>(p.s.:后文中,如果在同一层级的括号内,我们不必用括号规定的运算顺序为:$$\neg > \wedge = \vee > \to = \leftrightarrow$$)

[^优先级习惯0]: 此处的优先级顺序都是指在同一层括号内并且在同一层括号外(如果有内部的括号的话),如果有括号运算符规定运算顺序的话,按照括号内的运算顺序一定比括号外高来判断。

[^优先级习惯1]: 一种常用的优先级法则是合取运算符($$\wedge$$)的优先级高于析取运算符($$\vee$$)的优先级,但是这个规则不太好记,所以后文中会用括号来明细这两种运算符的运算顺序。
[^优先级习惯2]: 另一种常用的优先级法则是条件运算符($$\to$$)的优先级高于双条件运算符($$\leftrightarrow$$)的优先级,但是这个规则也不太好记,所以后文中会用括号来明细这两种运算符的运算顺序。

####1.1.6 逻辑运算与位运算

**位**:计算机用位表示信息。位:1/0,二进制。习惯上,用1表示真,用0表示假。(1 = True(T),0 = False(F))
**布尔变量**:布尔变量表示一个**取值只能为真或假**的变量。布尔变量可以用**位**来表示。

|真值|位|
|---||
|T|1|
|F|0|

**位运算**:计算机的位运算与逻辑连接词相对应。
**位串**:位串是0位或多位的序列。位串的长度就是其所含位的数目。

|逻辑连接词|位运算符|
|---|---|
|$$\neg $$|$$!$$(整体取反),$$~$$(按位取反)|
|$$\wedge$$|&(OR)|
|$$\vee$$|\|(AND)|
|$$\oplus$$|^(XOR)|

例如:
|x|y|x & y|x \| y|x ^ y|
|---|||||
|0|0|0|0|0|
|0|1|1|0|1|
|1|0|1|0|1|
|1|1|1|1|0|

|x|!x|~x|
|---|---|---|
|11111111|0|00000000|
|10001001|0|01110110|
|00000000|1|11111111|